%!TEX root = ../main.tex

\chapter{Component Symbol Design and Library Management}

Components in KiCad represent the electrical state of a single element in the schematic.
They also map the pin numbers of the element to a descriptive pin name.

In the schematic, we wire connections to pins based on their names and idiomatic positions.
The netlist is then generated using the pin number to map the schematic to the physical connections on the circuit board.

KiCad has a group of dedicated librarians who design, organize and curate a library of more than 10,000 symbols that are free to download at \url{https://kicad.github.io/symbols/}.  
Each symbol represents a distinct part.

\section{Libraries, Parts and Aliases}

In general, there are two types of symbols in the KiCad libraries: Atomic and Generic.
A Generic part is one for which there are many possible footprints and variations.
Examples of Generic parts are \textit{Resistors}, \textit{Capacitors}, \textit{BJTs} and similar parts.

Generic parts will typically not have footprints, footprint filters, manufacturer part numbers or similar identifying characteristics as there are many variations for each.

Atomic parts, on the other hand represent a specific manufacturer part.
These will typically have their associated datasheet, mpn and footprint assigned in the library.

Many components have alternate versions for small variations such as temperature or voltage compliance.
Where the pin numbers do not change between variations, we typically utilize \textit{aliases}.
An alias of a part is a component with the same pin number to pin name mapping but different part name and potentially different footprint, documentation and part description.

\section{KLC Design Guidelines}

The Kicad Library Convention (KLC) provides design guidelines for the creation of KiCad-like parts.
While nothing in the underlying KiCad code prevents you from breaking the convention, the KLC provides stylistic guidelines that will help your parts to work well with the existing libraries.
They are also required if and when you decide to submit your library parts for inclusion in the KiCad distribution.



\section{Graphic Items}

\section{Pins}
\subsection{Pin Names}
\subsection{Pin Numbers}
\subsection{Pin Types}
\subsection{Pin Stacking}
\subsection{Pin Table}

\section{Multi-unit Parts}
\subsection{De Morgan Parts}
\subsection{Heterogenous Parts}
\subsection{Homogeneous Styles}

\section{Field Properties}
\subsection{Footprints}
\subsection{Documentation}
\subsection{Keywords}

\section{Tips and Tricks for Common Mistakes}

\section{Symbol Design Examples}
\subsection{Example 1: Basic Symbol}
\subsection{Example 2: Heterogenous, Multi-unit Symbol}