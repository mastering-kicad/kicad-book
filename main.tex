\documentclass[nols,nofonts,notitlepage]{tufte-book}

\title{Mastering KiCad Version 5.0\thanks{Thanks to all the KiCad developers for their tireless efforts.}}
\author[Seth Hillbrand]{Seth~Hillbrand}
\publisher{}


\usepackage{fontenc}
\setmainfont[Renderer=Basic, Scale = 1.0, SmallCapsFont={Cormorant SC}]{Cormorant}
\setsansfont[Renderer=Basic, Scale=0.90]{Lato Regular}
\setmonofont[Renderer=Basic, Scale=0.80]{Cousine}

\usepackage{tikz}
\usepackage{microtype}
\usepackage{booktabs}
\usepackage{graphicx}
\setkeys{Gin}{width=\linewidth,totalheight=\textheight,keepaspectratio}
\graphicspath{{graphics/}}

\usepackage{geometry}
\geometry{  paperwidth = 7in,
			paperheight = 10in,
			left = 1in,
			top = 1in,
			headsep=2\baselineskip,
			textwidth=24pc,
			marginparsep=1pc,
			marginparwidth=10pc,
			textheight=44\baselineskip,
			headheight=\baselineskip
			 }

\usepackage{fancyvrb}
\fvset{fontsize=\normalsize}

% Hanging paren
\newcommand{\hangp}[1]{\makebox[0pt][r]{(}#1\makebox[0pt][l]{)}}

% Hanging asterix
\newcommand{\hangstar}{\makebox[0pt][l]{*}}

\usepackage{xspace}

\usepackage{ifluatex}
\ifluatex
  \usepackage{pdftexcmds}
  \makeatletter
  \let\pdfstrcmp\pdf@strcmp
  \let\pdffilemoddate\pdf@filemoddate
  \makeatother
\fi
\usepackage{svg}
\svgpath{{graphics/icons/}}
\setsvg{inkscapeexe="/Applications/Inkscape.app/Contents/Resources/bin/inkscape"}

\usepackage[most]{tcolorbox}
\newtcbox{\icontcbox}{colback=orange!10, colframe=orange!50,left=0mm,right=0mm,top=0mm,bottom=0mm,height=6ex,width=6ex,valign=center,arc=0mm,boxrule=0.5pt}

% Prints the month name (e.g., January) and the year (e.g., 2008)
\newcommand{\monthyear}{%
  \ifcase\month\or January\or February\or March\or April\or May\or June\or
  July\or August\or September\or October\or November\or
  December\fi\space\number\year
}

\setlength{\parskip}{1em}

\usepackage{menukeys}

\newcommand{\blankpage}{\newpage\hbox{}\thispagestyle{empty}\newpage}

\newcommand\iconbox[1]{\icontcbox{\includesvg[width=4ex]{#1}}}
\newcommand\iconstart[1]{\noindent\smash{\lower4ex\hbox{\llap{\iconbox{#1}}}\hskip+0.1em}
  \parshape=3 1.25em \dimexpr\hsize-1.75em 1.75em \dimexpr\hsize-1em 0.5pt \hsize}
\newcommand\iconstarts[2]{\noindent\smash{\lower4ex\hbox{\llap{\iconbox{#1}}}\lower10ex\hbox{\llap{\iconbox{#2}}}\hskip+0.1em}
  \parshape=6 1.25em \dimexpr\hsize-1.75em 1.75em \dimexpr\hsize-1.75em 1.75em \dimexpr\hsize-1.75em 1.75em \dimexpr\hsize-2em 0pt \hsize 0pt \hsize}
  
% Macros for typesetting the documentation
\newcommand{\hlred}[1]{\textcolor{Maroon}{#1}}% prints in red
\newcommand{\hangleft}[1]{\makebox[0pt][r]{#1}}
\newcommand{\hairsp}{\hspace{1pt}}% hair space
\newcommand{\hquad}{\hskip0.5em\relax}% half quad space
\newcommand{\TODO}{\textcolor{red}{\bf TODO!}\xspace}
\newcommand{\na}{\quad--}% used in tables for N/A cells

\newcommand{\tuftebs}{\symbol{'134}}% a backslash in tt type in OT1/T1
\newcommand{\doccmdnoindex}[2][]{\texttt{\tuftebs#2}}% command name -- adds backslash automatically (and doesn't add cmd to the index)
\newcommand{\doccmddef}[2][]{%
  \hlred{\texttt{\tuftebs#2}}\label{cmd:#2}%
  \ifthenelse{\isempty{#1}}%
    {% add the command to the index
      \index{#2 command@\protect\hangleft{\texttt{\tuftebs}}\texttt{#2}}% command name
    }%
    {% add the command and package to the index
      \index{#2 command@\protect\hangleft{\texttt{\tuftebs}}\texttt{#2} (\texttt{#1} package)}% command name
      \index{#1 package@\texttt{#1} package}\index{packages!#1@\texttt{#1}}% package name
    }%
}% command name -- adds backslash automatically
\newcommand{\doccmd}[2][]{%
  \texttt{\tuftebs#2}%
  \ifthenelse{\isempty{#1}}%
    {% add the command to the index
      \index{#2 command@\protect\hangleft{\texttt{\tuftebs}}\texttt{#2}}% command name
    }%
    {% add the command and package to the index
      \index{#2 command@\protect\hangleft{\texttt{\tuftebs}}\texttt{#2} (\texttt{#1} package)}% command name
      \index{#1 package@\texttt{#1} package}\index{packages!#1@\texttt{#1}}% package name
    }%
}% command name -- adds backslash automatically
\newcommand{\docopt}[1]{\ensuremath{\langle}\textrm{\textit{#1}}\ensuremath{\rangle}}% optional command argument
\newcommand{\docarg}[1]{\textrm{\textit{#1}}}% (required) command argument
\newenvironment{docspec}{\begin{quotation}\ttfamily\parskip0pt\parindent0pt\ignorespaces}{\end{quotation}}% command specification environment
\newcommand{\docenv}[1]{\texttt{#1}\index{#1 environment@\texttt{#1} environment}\index{environments!#1@\texttt{#1}}}% environment name
\newcommand{\docenvdef}[1]{\hlred{\texttt{#1}}\label{env:#1}\index{#1 environment@\texttt{#1} environment}\index{environments!#1@\texttt{#1}}}% environment name
\newcommand{\docpkg}[1]{\texttt{#1}\index{#1 package@\texttt{#1} package}\index{packages!#1@\texttt{#1}}}% package name
\newcommand{\doccls}[1]{\texttt{#1}}% document class name
\newcommand{\docclsopt}[1]{\texttt{#1}\index{#1 class option@\texttt{#1} class option}\index{class options!#1@\texttt{#1}}}% document class option name
\newcommand{\docclsoptdef}[1]{\hlred{\texttt{#1}}\label{clsopt:#1}\index{#1 class option@\texttt{#1} class option}\index{class options!#1@\texttt{#1}}}% document class option name defined
\newcommand{\docmsg}[2]{\bigskip\begin{fullwidth}\noindent\ttfamily#1\end{fullwidth}\medskip\par\noindent#2}
\newcommand{\docfilehook}[2]{\texttt{#1}\index{file hooks!#2}\index{#1@\texttt{#1}}}
\newcommand{\doccounter}[1]{\texttt{#1}\index{#1 counter@\texttt{#1} counter}}

% Generates the index
\usepackage{makeidx}


\makeindex

\begin{document}
\frontmatter
\blankpage
{
  \let\allcaps=\relax
  \maketitle
}

\newpage
\begin{fullwidth}
~\vfill
\thispagestyle{empty}
\setlength{\parindent}{0pt}
\setlength{\parskip}{\baselineskip}
Copyright \copyright\ \the\year\ \thanklessauthor

\par\smallcaps{Published by \thanklesspublisher}

\par\smallcaps{https://mastering-kicad.github.io/v5}

\par This work is licensed under the Creative Commons Attribution 4.0 International License (the ``License''). 
To view a copy of this license, visit \url{http://creativecommons.org/licenses/by/4.0/} or send a letter to Creative Commons, PO Box 1866, Mountain View, CA 94042, USA. 

\par Code samples included in this book are licensed under the terms of the GNU General Public License as published by the Free Software Foundation, either version 3 of the License, or (at your option) any later version. 
To view a copy of this license, visit \url{https://www.gnu.org/licenses/gpl-3.0.html}.\index{license}

\par\textit{First printing, \monthyear}
\end{fullwidth}

\tableofcontents

\listoffigures

\listoftables

\cleardoublepage
~\vfill
\begin{doublespace}
\noindent\fontsize{18}{22}\selectfont\itshape
\nohyphenation
Dedicated to people \TODO Insert ded.
\end{doublespace}
\vfill
\vfill


\cleardoublepage
\chapter*{Introduction}

\newthought{A project like KiCad} does not have a single or even primary person who is responsible for the plurality of the work.
Instead, there are many, many individuals, spread across the globe who come together over mailing lists and bug trackers to create something as a whole.
People will often drift into and out of such endeavors as their time and inclination permits.
But they leave a last, indelible and -- hopefully -- positive impact on the codebase, user experience, libraries, documentation and artwork that make up the amazing ecosystem of KiCad.

That said, a few individuals' contributions help to ensure that the ecosystem exists in the first place, that contributors are encouraged and that activity is focused where it will do the most good.
KiCad is fortunate to have (and have had) the contributions of Jean-Pierre Charras, whose work has been the foundation of KiCad and whose continued contribution to its development is an example of selfless dedication.
In addition, Dick Hollenbeck and his employer SoftPLC contributed a substantial fraction of the KiCad internals and structure.
Wayne Stambaugh who, in his role as project leader, has guided the KiCad team since 2014 and provided a much-needed steadying hand as the project developed into what it is today.
In addition to these three, there are many, many more dedicated, collaborative developers working to make KiCad stronger.

Part of making KiCad stronger is introducing it to a wider audience and, hopefully, reducing some of the headaches that accompany learning a new piece of software.
That is what I hope this book will provide.
This book is meant to supplement rather than supplant the official KiCad documentation\sidenote{\url{http://kicad-pcb.org/help/documentation/}}.
While intended aimed primarily at a University audience learning KiCad and PCB design simultaneously, I hope that it also provides a helpful reference for regular KiCad users looking to improve their usage patterns.

\mainmatter

%!TEX root = ../main.tex

\chapter{Introduction to KiCad}
\label{ch:intro-kicad}



\section{Circuit Design}


\section{Prototyping}

\section{Printed Circuit Boards}

\section{Installing KiCad}
\subsection{Linux}
\subsection{MacOS}
\subsection{Windows}

\section{Installing Libraries and Models}

\section{First Run}
\subsection{Setup Paths}
\subsection{Setup Symbol Library Table}
\subsection{Setup Footprint Library Table}

\section{Upgrading from Version 4}

\section{Opening Example Projects}

\section{Summary}
%%!TEX root = ../main.tex

\chapter{KiCad Design Workflow}
\section{Project Initialization}
\section{Component Selection}
\section{Schematic Layout}
\section{Electrical Rules Check}
\section{Footprint Association}
\section{Circuit Board Layout}
\section{Design Rules Check}
\section{Generating Design Files}
\subsection{Gerber Files}
\subsection{IPC Files}
\subsection{XY Placement Files}
%%!TEX root = ../main.tex

\chapter{KiCad Project Structure}
\label{ch:project-structure}

\section{Using Git}
\subsection{Version Control Basics}
\subsection{Local vs. Remote}
\subsection{Shallow Copies}
\subsection{Personal Libraries}

\section{Building Robust Projects}

\section{Template Design}

\section{KiCad Component Libraries Explained}
\subsection{Symbol Library Management}
\subsection{Footprint Library Management}

\section{Project Files and Sub-files}
%!TEX root = ../main.tex

\chapter{Component Symbol Design and Library Management}

\section{Libraries, Parts and Aliases}

\section{KLC Design Guidelines}

\section{Graphic Items}

\section{Pins}
\subsection{Pin Names}
\subsection{Pin Numbers}
\subsection{Pin Types}
\subsection{Pin Stacking}
\subsection{Pin Table}

\section{Multi-unit Parts}
\subsection{De Morgan Parts}
\subsection{Heterogenous Parts}
\subsection{Homogeneous Styles}

\section{Field Properties}
\subsection{Footprints}
\subsection{Documentation}
\subsection{Keywords}

\section{Tips and Tricks for Common Mistakes}

\section{Symbol Design Examples}
\subsection{Example 1: Basic Symbol}
\subsection{Example 2: Heterogenous, Multi-unit Symbol}
%!TEX root = ../main.tex

\chapter{Schematic Editor}

The first step in every project is schematic capture.
If this term is unfamiliar, we are referring to the process of laying out the components and connections between them in logical, well-defined fashion.
You may have only the barest idea of the circuit aspects or you may have the major aspects already sketched on paper;
whatever your situation, transferring the schematic into KiCad begins with the Schematic Editor.

Before we dive into the simplistic aspects of the schematic editor, it will pay us dividends later to properly understand how the schematic editor views more complex schematics.
This will allow you to structure the schematic in your mind more closely to how Schematic Editor expects.

\section{Hierarchical Schematics Explained}

The basic, underlying paradigm of the Schematic Editor is the idea of \textit{hierarchical schematics}.
If the overall schematic represents the full circuit that will exist on the circuit board at the end of production, then the hierarchy provides a logical grouping of similar component sets.
This allows you to reuse common subsets throughout your design.
The hierarchy also explicitly structures your schematics as a top-down tree.

At the root of the tree, is the first schematic sheet you open, usually with the same name as your project and the extension `\textbf{.sch}'.
Off of this root, you can create \textit{sub-sheets} that are placed in the root sheet.
The sub-sheet has both a ``File name'' and a ``Sheet name''.
The File name is the name of the sub-sheet on the underlying filesystem.
This is the name you will see if you open your project folder outside of KiCad.
The Sheet name is the internal reference of the specific \textit{instance} of the sub-sheet in KiCad.
\begin{figure}
	\includegraphics{chapter5/subsheet.png}
	\caption[Sub-Sheets]{The subsheet dialog specifies the file name, sheet name and a unique timestamp.}
\end{figure}

\section{Understanding the Schematic Editor}
\subsection{Relationship to Symbol Libraries}
\subsection{Project Settings}

\section{Adding Parts}

\section{Connections}
\subsection{Wires}
\subsection{Busses}
\subsection{Labels}
\subsection{Junctions}
\subsection{No Connects}
\subsection{Hierarchical Pins}

\section{Electrical Rules Check}

\section{Graphical Elements}

\section{Templates}

\section{Schematic Editor Examples}
\subsection{Example 1: Single Page Schematic}
\subsection{Example 2: Hierarchical Schematic}
%%!TEX root = ../main.tex

\chapter{Footprint Design with KiCad}

\section{Footprint Types}
\subsection{Through hole}
\subsection{SMD}
\subsection{Virtual}

\section{Layers}

\section{Pins and Pads}
\subsection{Pad Shapes}
\subsection{Custom Shapes}
\subsection{Masks}
\subsection{Special Pins}

\section{Clearance Settings}

\section{Thermal Relief}

\section{3d Settings}

\section{Creating Footprints using Generators}

\section{Footprint Design Examples}
\subsection{Example 1: QFN with Exposed Pad}
\subsection{Example 2: High-Density, Through-hole Connector}
\subsection{Example 2: Board-Edge Connector}
%%!TEX root = ../main.tex

\chapter{Printed Circuit Board Layout}
\label{ch:pcbnew}

\section{Basic Board Settings}

\section{Board Layers}
\subsection{Layer Pairs}
\subsection{Special Layers}
\subsection{User Layers}
\subsection{Customizing your Layers}

\section{Adding Components}
\subsection{Updating from Schematic}
\subsection{Placement and Rotation}
\subsection{Adding Dangling Components}

\section{Connecting Components}
\subsection{Netclasses}
\subsection{Wires}
\subsection{Copper Zones}

\section{Copper Zones}
\subsection{Spacing}
\subsection{Connections}
\subsection{Solid vs. Thermals}
\subsection{Pads and Antipads}
\subsection{Fill Settings}
\subsection{Keepout Zones}

\section{Stitching Zones}

\section{Revising}
\subsection{Locking Items}
\subsection{Push and Shove}
\subsection{Updating from Schematic}
\subsection{Changing Footprints}
\subsection{Trace cleanup}

\section{Special Topics}
\subsection{Designing for Heat Distribution}
\subsection{Transmission Line Trace Length}
\subsection{Blind and Buried Vias}
\subsection{Layout arrays}

\section{Layout Principles}

\section{PCB Design Examples}
\subsection{Example 1: Single Layer Photodiode Amplifier}
\subsection{Example 2: Dual Layer, Gated Delay Shaping Amplifier}
\subsection{Example 3: Mixed Domain Analog/Digital Reward Controller}

%%!TEX root = ../main.tex

\chapter{Signal Integrity Design}

\section{Circuit Design}
\subsection{Noise Issues}
\subsection{Frequency Limits}
\subsection{Grounds}

\section{PCB Routing Design}
\subsection{Avoiding Antennas}
\subsection{External vs. Internal Noise Sources}
\subsection{Inductive Coupling}
\subsection{Capacitive Coupling}
\subsection{Trace width and spacing}
\subsection{Current limits}
\subsection{Voltage limits}
\subsection{Special Requirements}
\subsection{Trace angles}

\section{Grounding}
\subsection{Ground Planes and Stitching}
\subsection{Split Grounds}

\section{PCB Layout Design}
\subsection{Component Grouping}
\subsection{Layer Stackups}
\subsection{Bypass capacitors}
\subsection{Sizing}
\subsection{Type}
\subsection{Placement}

\section{External Connections}
\subsection{Power Connections}
\subsection{Signal Connections}
\subsection{Antennae}

\section{Signal Integrity Examples}
\subsection{Example 1: Revising a Design for Signal Integrity}
\subsection{Example 2: Protecting Your Circuit}

%%!TEX root = ../main.tex

\chapter{Design for Manufacturing, Design for Testing}

\section{Standards}
\subsection{IPC}
\subsection{IEC}
\subsection{JEDEC}
\subsection{MILSPEC}

\section{Tolerances}
\subsection{Trace - Trace}
\subsection{Trace - Pad}
\subsection{Pad - Hole}
\subsection{Annular ring size}
\subsection{Drill Sizes}

\section{Panelling}
\subsection{Standard Panel Sizes}
\subsection{Tooling Areas}
\subsection{Cores and Prepregs}
\subsection{Copper Cladding Weights and Thicknesses}

\section{Soldermask}
\subsection{Vias and tenting}

\section{DNP placements}
\section{Test points}
\section{Teardrops}
\section{Thermal Relief}
\section{Component Orientation and Spacing}
\section{Soldering Processes}
\subsection{Hand Soldering}
\subsection{Wave Soldering}
\subsection{Reflow Soldering}
%%!TEX root = ../main.tex

\chapter{Circuit Board Tweaks and Revisioning}
\section{Resizing}

\section{Swapping Parts}

\section{Reducing Complexity}
\subsection{Layer Reduction}
\subsection{Component Count Reduction}
\subsection{Routing Simplification}
\subsection{Via Minimization}
%!TEX root = ../main.tex

\chapter{Project Design Examples}
\label{ch:examples}

\section{Example 1: 4-Channel Communications Card}
\subsection{Schematic Setup}
\subsection{Design Requirements}
\section{Example 2: Nanosecond LED Pulser}
\subsection{Schematic Setup}
\subsection{Design Requirements}
\section{Example 3: 8-Channel Ultra low-noise Shaping Amplifier}
\subsection{Schematic Setup}
\subsection{Design Requirements}
%%!TEX root = ../main.tex
\chapter{SPICE Modeling in KiCad}
\label{ch:spice}

\section{Getting SPICE working}
\section{Adding Sources}
\section{Simulating Analog Systems}
\section{Simulating Transmission Lines}
\section{Including Vendor Models}


\backmatter

\bibliography{kicad}
\bibliographystyle{plainnat}


\printindex

\end{document}

